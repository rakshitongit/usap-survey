%%%%%%%%%%%%%%%%%%%%%%%%%%%%%%%%%%%%%%%%%%%%%%%%%%%%%%%%%%%%%%%%%%%%%%%%%%%%%%%%
% Template for USENIX papers.
%
% History:
%
% - TEMPLATE for Usenix papers, specifically to meet requirements of
%   USENIX '05. originally a template for producing IEEE-format
%   articles using LaTeX. written by Matthew Ward, CS Department,
%   Worcester Polytechnic Institute. adapted by David Beazley for his
%   excellent SWIG paper in Proceedings, Tcl 96. turned into a
%   smartass generic template by De Clarke, with thanks to both the
%   above pioneers. Use at your own risk. Complaints to /dev/null.
%   Make it two column with no page numbering, default is 10 point.
%
% - Munged by Fred Douglis <douglis@research.att.com> 10/97 to
%   separate the .sty file from the LaTeX source template, so that
%   people can more easily include the .sty file into an existing
%   document. Also changed to more closely follow the style guidelines
%   as represented by the Word sample file.
%
% - Note that since 2010, USENIX does not require endnotes. If you
%   want foot of page notes, don't include the endnotes package in the
%   usepackage command, below.
% - This version uses the latex2e styles, not the very ancient 2.09
%   stuff.
%
% - Updated July 2018: Text block size changed from 6.5" to 7"
%
% - Updated Dec 2018 for ATC'19:
%
%   * Revised text to pass HotCRP's auto-formatting check, with
%     hotcrp.settings.submission_form.body_font_size=10pt, and
%     hotcrp.settings.submission_form.line_height=12pt
%
%   * Switched from \endnote-s to \footnote-s to match Usenix's policy.
%
%   * \section* => \begin{abstract} ... \end{abstract}
%
%   * Make template self-contained in terms of bibtex entires, to allow
%     this file to be compiled. (And changing refs style to 'plain'.)
%
%   * Make template self-contained in terms of figures, to
%     allow this file to be compiled. 
%
%   * Added packages for hyperref, embedding fonts, and improving
%     appearance.
%   
%   * Removed outdated text.
%
%%%%%%%%%%%%%%%%%%%%%%%%%%%%%%%%%%%%%%%%%%%%%%%%%%%%%%%%%%%%%%%%%%%%%%%%%%%%%%%%

%%% Minor updates for SOUPS 2019 by Michelle Mazurek

\documentclass[letterpaper,twocolumn,10pt]{article}
\usepackage{usenix2021_SOUPS}

% to be able to draw some self-contained figs
\usepackage{tikz}
\usepackage{amsmath}

% inlined bib file
\usepackage{filecontents}

%-------------------------------------------------------------------------------
\begin{filecontents}{\jobname.bib}
%-------------------------------------------------------------------------------
@inproceedings {197318,
author = {Shrirang Mare and Mary Baker and Jeremy Gummeson},
title = {A Study of Authentication in Daily Life},
booktitle = {Twelfth Symposium on Usable Privacy and Security (SOUPS 2016)},
year = {2016},
isbn = {978-1-931971-31-7},
address = {Denver, CO},
pages = {189--206},
url = {https://www.usenix.org/conference/soups2016/technical-sessions/presentation/mare},
publisher = {USENIX Association},
month = jun,
}
@article{lillegaard,
author = {Lillegaard, Inger and Løken, E and Andersen, Lene},
year = {2007},
month = {02},
pages = {61-8},
title = {Relative validation of a pre-coded food diary among children, under-reporting varies with reporting day and time of the day},
volume = {61},
journal = {European journal of clinical nutrition},
doi = {10.1038/sj.ejcn.1602487}
}
\end{filecontents}

%-------------------------------------------------------------------------------
\begin{document}
%-------------------------------------------------------------------------------

%don't want date printed
\date{}

% make title bold and 14 pt font (Latex default is non-bold, 16 pt)
\title{\Large \bf A Replication Study of Authentication in Daily life of the University Students}

% if you leave this blank it will default to a possibly ugly attempt 
% to make the contents of the \author command below into a string
\def\plainauthor{Author name(s) for PDF metadata. Don't forget to anonymize for submission!}

%for single author (just remove % characters)
\author{
{\rm Rakshit Bhat}\\
University Paderborn
\and
{\rm Nidhi Nadig}\\
University Paderborn
% copy the following lines to add more authors
\and
{\rm Subramanya Padmaraja Setty}\\
University Paderborn
\and
{\rm Mayank Kapadi}\\
University Paderborn
\and
{\rm Pritom Touhid Hossain}\\
University Paderborn
\and
{\rm Showmik Md Jannatul Baki}\\
University Paderborn
} % end author

\maketitle
\thecopyright

%-------------------------------------------------------------------------------
\begin{abstract}
%-------------------------------------------------------------------------------
We are surveying 20 people to determine the usability of the daily authentication techniques on different authentication devices. We determine various target devices, including phones, PCs, websites, doors, Tablets, etc. with various authenticators like passwords, PINs, physical keys, Physical tokens, fingerprints, etc. Our research focuses on determining how stressful managing different authentication techniques are and improving these techniques to benefit the users and the companies.
\end{abstract}


%-------------------------------------------------------------------------------
\section{Introduction}
%-------------------------------------------------------------------------------
Many of us carry a lot of authentication devices with us in our daily life. These devices include a car key, house key, corporate badge, bike key, RSA token, bus pass, credit card, driver's license, ATM card \cite{197318}. We also remember the passwords and use fingerprints and faceiDs to unlock the devices. These are called authenticators, tools for authentications or confirming the user's identity. By proving that they own and are in control of an authenticator, a person authenticates to a computer system or application. \\
\textbf{Types of authentications:} We formalised a few most common authentication techniques that users use daily. These include security questions, Passwords, Pins, SMS voice or email OTPs, Push notifications, Physical tokens, and Biometrics.
\begin{enumerate}
    \item \textbf{Security Questions:} These are some personal questions which are being asked by the Software to identify the users and verify them.
    \item \textbf{Passwords:} Softwares use classical passwords, secret words, or a group of words to verify the user. 
    \item \textbf{Pins:} These are usually numbers used in mobiles or laptops as an alternative to the passwords for quick login. 
    \item \textbf{OTPs:} OTP or One-time password means Software will send a password on your devices like mobiles or laptops via SMS or emails.
    \item \textbf{Push notifications:} This is one of the latest forms of the authentication technique in which the users need first to configure their mobile devices with the Software. Later while authenticating the user, the Software will send a push notification to the registered mobile device wherein the user can confirm and log in. 
    \item \textbf{Physical tokens:} These are physical keys encrypted using cryptographic algorithms and randomly generated each time the user wants to authenticate.
    \item \textbf{Biometrics:} These include fingerprints, voice recognition, face IDs etc., specially designed to recognise the user uniquely.
\end{enumerate}
Today, it has become very stressful for users with so many different types of authentication techniques available.  Users have questions like Whom to trust or which of these authentication types is more secure. There are some problems with these authentication techniques; if you enable 2-factor authentication and lose the device responsible for that, you will not be able to authenticate further. Witty et al. \cite{} concluded from their research that it takes around \$50 - \$150 to resolve issues and reset passwords. We raise a few questions to determine the usability of the authentication. 
\begin{enumerate}
    \item What are the users' various types of authentication techniques, and how much of a burden is it?
    \item How failure-prone are the different types of authentication? 
    \item Does the general public concur on the types of authentication they prefer or not?
\end{enumerate}

To tackle these queries, we conducted a wearable digital diary user survey of 20 people, which included university students, to better understand the user authentication burden. This will be explained in detailed in section \ref{sec:method}.


%-------------------------------------------------------------------------------
\section{Related Work}
%-------------------------------------------------------------------------------

Footnotes should be places after punctuation characters, without any
spaces between said characters and footnotes, like so.%
\footnote{Github URL: https://github.com/rakshitongit/usap-survey} And some embedded literal code may
look as follows.

\begin{verbatim}
int main(int argc, char *argv[]) 
{
    return 0;
}
\end{verbatim}

Now we're going to cite somebody. Watch for the cite tag. Here it
comes. Arpachi-Dusseau and Arpachi-Dusseau co-authored an excellent OS
book, which is also really funny~\cite{arpachiDusseau18:osbook}, and
Waldspurger got into the SIGOPS hall-of-fame due to his seminal paper
about resource management in the ESX hypervisor~\cite{waldspurger02}.

The tilde character (\~{}) in the tex source means a non-breaking
space. This way, your reference will always be attached to the word
that preceded it, instead of going to the next line.

And the 'cite' package sorts your citations by their numerical order
of the corresponding references at the end of the paper, ridding you
from the need to notice that, e.g, ``Waldspurger'' appears after
``Arpachi-Dusseau'' when sorting references
alphabetically~\cite{waldspurger02,arpachiDusseau18:osbook}. 

It'd be nice and thoughtful of you to include a suitable link in each
and every bibtex entry that you use in your submission, to allow
reviewers (and other readers) to easily get to the cited work, as is
done in all entries found in the References section of this document.

Now we're going take a look at Section~\ref{sec:figs}, but not before
observing that refs to sections and citations and such are colored and
clickable in the PDF because of the packages we've included.

%-------------------------------------------------------------------------------
\section{Methodology}
\label{sec:method}
%-------------------------------------------------------------------------------


%---------------------------

%% %---------------------------

In this section, we will discuss the Authentication event (what kind of data will the participants log) and the Digital diary (how will the participant log the data)

\subsection{Authentication event}
The participants must log the authentication data each time they do an authentication event. To define the term authentication event, we did some research on the previous researchers \cite{197318} and replicated it. Mare et al. \cite{197318} have defined the authentication event as where the participants must verify that they are the appropriate people to access typical resources or services through something they are or something they have. Examples include unlocking a mobile or tablet device, a house door, login to a website or banking transactions, etc. Although, we do not consider lock or re-lock as an authentication event. We now define an authentication target as a device, a resource or service through which the user can request access, and an authenticator as proof that the user needs to provide to gain access. Consider an example of users unlocking their mobile phones with their fingerprints. Here mobile phones are authentication target devices, and fingerprint/biometrics are authenticators. Similarly, opening a house door with a key, the door is the authentication target device, and the house key is the authenticator. Some more examples of authenticators and authentication target. \\ \\
\noindent
\textbf{Authentication Targets:} Computers (Laptop, desktops), Mobile Phone, Tablet (also e-readers), Website (also online websites or any software), Door, Car, ATM, Bicycle (also motorcycle), Phone payment, Card payment, Bank check, Locker (also locked drawers), and Other.
\\ \\
\noindent
\textbf{Authenticators:} Password (locker combinations, etc.), PINS, Biometrics (Fingerprint, Face biometrics, Voice biometrics), Card (ID cards, credit cards, badges), Physical tokens (Certificate (PKI)), Mouse click (where the participant has to click to authenticate, e.g., to request autofill with a password manager), Lock key (physical key), Signature, 2-Factor authenticators, and Other%
\footnote{These are general authenticators and authentication target devices which are being used by many users.}.

In the survey, we are also interested in knowing the authentication location and the time the authentication takes place. We are also interested in the number of attempts before successful authentication and the convenience of the authentication events. Therefore, our authentication event is a tuple consisting of \{\textit{authentication-location, authentication-time, authentication target, authenticator, authentication-success, authentication-convenience}\}
\subsection{Digital Diary}
\begin{figure}
\begin{center}
  \includegraphics[width=150]{images/profile.png}
\end{center}
\caption{\label{fig:app-profile} Mobile application - User Profile}
\end{figure}
\begin{figure}
\begin{center}
  \includegraphics[width=150]{images/main-survey.png}
\end{center}
\caption{\label{fig:app-main-survey} Mobile application - Main Survey}
\end{figure}
\begin{figure}
\begin{center}
  \includegraphics[width=150]{images/daily-survey.png}
\end{center}
\caption{\label{fig:app-daily-survey} Mobile application - Daily Survey}
\end{figure}
Lillegaard et al. \cite{lillegaard} suggested that digital dairies are a practical way to immediately and easily log events. Digital diaries are especially useful for events like authentication, which happen repeatedly and pulling out a paper notebook and pen or using a physical journal is difficult. Therefore, we developed a mobile application%
\footnote{Github URL: https://github.com/rakshitongit/usap-survey} that is easy to use and download for android users. For the other users, we deployed the webview%
\footnote{URL: https://n5lis62ifu.appflowapp.com/} of the mobile app. Most of the participants found the app convenient to log the data. As shown in Figure \ref{fig:app-profile}, the participants must first fill up some of their personal data. As soon as they fill in this information, the participants are permitted to start with the survey. The survey page contains various types of questions (see Figure \ref{fig:app-main-survey}). The participant cannot submit unless all the questions are answered. The mobile application automatically collects the location (latitude and longitude) and the timestamp at which the participant has logged the data. In the neighbouring tab (see Figure \ref{fig:app-daily-survey}), the participant can find the daily questions, which the participant needs to fill in for the feedback and the qualitative data analysis. 
\subsection{Methodology successes and failures}
\subsection{Study procedure}
Recruitment and enrollment
Pre-logging interview
Post-logging interview
\subsection{Methodology successes and failures}

\section{LIMITATIONS}
\section{FINDINGS}
Results - Qualitative and quantitative analysis
\subsection{Authentication burden}
%-----------------------------------

People often use \texttt{pdflatex} these days for creating pdf-s from
tex files via the shell. And \texttt{bibtex}, of course. Works for us.

%-------------------------------------------------------------------------------
\section*{Acknowledgments}
%-------------------------------------------------------------------------------

The USENIX latex style is old and very tired, which is why
there's no \textbackslash{}acks command for you to use when
acknowledging. Sorry.


%-------------------------------------------------------------------------------
\bibliographystyle{plain}
\bibliography{\jobname}

%%%%%%%%%%%%%%%%%%%%%%%%%%%%%%%%%%%%%%%%%%%%%%%%%%%%%%%%%%%%%%%%%%%%%%%%%%%%%%%%
\end{document}
%%%%%%%%%%%%%%%%%%%%%%%%%%%%%%%%%%%%%%%%%%%%%%%%%%%%%%%%%%%%%%%%%%%%%%%%%%%%%%%%

%%  LocalWords:  endnotes includegraphics fread ptr nobj noindent
%%  LocalWords:  pdflatex acks
